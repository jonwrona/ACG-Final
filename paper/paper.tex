\documentclass[journal, a4paper]{IEEEtran}

% some very useful LaTeX packages include:

%\usepackage{cite}      % Written by Donald Arseneau
                        % V1.6 and later of IEEEtran pre-defines the format
                        % of the cite.sty package \cite{} output to follow
                        % that of IEEE. Loading the cite package will
                        % result in citation numbers being automatically
                        % sorted and properly "ranged". i.e.,
                        % [1], [9], [2], [7], [5], [6]
                        % (without using cite.sty)
                        % will become:
                        % [1], [2], [5]--[7], [9] (using cite.sty)
                        % cite.sty's \cite will automatically add leading
                        % space, if needed. Use cite.sty's noadjust option
                        % (cite.sty V3.8 and later) if you want to turn this
                        % off. cite.sty is already installed on most LaTeX
                        % systems. The latest version can be obtained at:
                        % http://www.ctan.org/tex-archive/macros/latex/contrib/supported/cite/

\usepackage{graphicx}   % Written by David Carlisle and Sebastian Rahtz
                        % Required if you want graphics, photos, etc.
                        % graphicx.sty is already installed on most LaTeX
                        % systems. The latest version and documentation can
                        % be obtained at:
                        % http://www.ctan.org/tex-archive/macros/latex/required/graphics/
                        % Another good source of documentation is "Using
                        % Imported Graphics in LaTeX2e" by Keith Reckdahl
                        % which can be found as esplatex.ps and epslatex.pdf
                        % at: http://www.ctan.org/tex-archive/info/

%\usepackage{psfrag}    % Written by Craig Barratt, Michael C. Grant,
                        % and David Carlisle
                        % This package allows you to substitute LaTeX
                        % commands for text in imported EPS graphic files.
                        % In this way, LaTeX symbols can be placed into
                        % graphics that have been generated by other
                        % applications. You must use latex->dvips->ps2pdf
                        % workflow (not direct pdf output from pdflatex) if
                        % you wish to use this capability because it works
                        % via some PostScript tricks. Alternatively, the
                        % graphics could be processed as separate files via
                        % psfrag and dvips, then converted to PDF for
                        % inclusion in the main file which uses pdflatex.
                        % Docs are in "The PSfrag System" by Michael C. Grant
                        % and David Carlisle. There is also some information
                        % about using psfrag in "Using Imported Graphics in
                        % LaTeX2e" by Keith Reckdahl which documents the
                        % graphicx package (see above). The psfrag package
                        % and documentation can be obtained at:
                        % http://www.ctan.org/tex-archive/macros/latex/contrib/supported/psfrag/

%\usepackage{subfigure} % Written by Steven Douglas Cochran
                        % This package makes it easy to put subfigures
                        % in your figures. i.e., "figure 1a and 1b"
                        % Docs are in "Using Imported Graphics in LaTeX2e"
                        % by Keith Reckdahl which also documents the graphicx
                        % package (see above). subfigure.sty is already
                        % installed on most LaTeX systems. The latest version
                        % and documentation can be obtained at:
                        % http://www.ctan.org/tex-archive/macros/latex/contrib/supported/subfigure/

\usepackage{url}        % Written by Donald Arseneau
                        % Provides better support for handling and breaking
                        % URLs. url.sty is already installed on most LaTeX
                        % systems. The latest version can be obtained at:
                        % http://www.ctan.org/tex-archive/macros/latex/contrib/other/misc/
                        % Read the url.sty source comments for usage information.

%\usepackage{stfloats}  % Written by Sigitas Tolusis
                        % Gives LaTeX2e the ability to do double column
                        % floats at the bottom of the page as well as the top.
                        % (e.g., "\begin{figure*}[!b]" is not normally
                        % possible in LaTeX2e). This is an invasive package
                        % which rewrites many portions of the LaTeX2e output
                        % routines. It may not work with other packages that
                        % modify the LaTeX2e output routine and/or with other
                        % versions of LaTeX. The latest version and
                        % documentation can be obtained at:
                        % http://www.ctan.org/tex-archive/macros/latex/contrib/supported/sttools/
                        % Documentation is contained in the stfloats.sty
                        % comments as well as in the presfull.pdf file.
                        % Do not use the stfloats baselinefloat ability as
                        % IEEE does not allow \baselineskip to stretch.
                        % Authors submitting work to the IEEE should note
                        % that IEEE rarely uses double column equations and
                        % that authors should try to avoid such use.
                        % Do not be tempted to use the cuted.sty or
                        % midfloat.sty package (by the same author) as IEEE
                        % does not format its papers in such ways.

\usepackage{amsmath}    % From the American Mathematical Society
                        % A popular package that provides many helpful commands
                        % for dealing with mathematics. Note that the AMSmath
                        % package sets \interdisplaylinepenalty to 10000 thus
                        % preventing page breaks from occurring within multiline
                        % equations. Use:
%\interdisplaylinepenalty=2500
                        % after loading amsmath to restore such page breaks
                        % as IEEEtran.cls normally does. amsmath.sty is already
                        % installed on most LaTeX systems. The latest version
                        % and documentation can be obtained at:
                        % http://www.ctan.org/tex-archive/macros/latex/required/amslatex/math/



% Other popular packages for formatting tables and equations include:

%\usepackage{array}
% Frank Mittelbach's and David Carlisle's array.sty which improves the
% LaTeX2e array and tabular environments to provide better appearances and
% additional user controls. array.sty is already installed on most systems.
% The latest version and documentation can be obtained at:
% http://www.ctan.org/tex-archive/macros/latex/required/tools/

% V1.6 of IEEEtran contains the IEEEeqnarray family of commands that can
% be used to generate multiline equations as well as matrices, tables, etc.

% Also of notable interest:
% Scott Pakin's eqparbox package for creating (automatically sized) equal
% width boxes. Available:
% http://www.ctan.org/tex-archive/macros/latex/contrib/supported/eqparbox/

% *** Do not adjust lengths that control margins, column widths, etc. ***
% *** Do not use packages that alter fonts (such as pslatex).         ***
% There should be no need to do such things with IEEEtran.cls V1.6 and later.


% Your document starts here!
\begin{document}

% Define document title and author
    \title{A Proposal for Rendering Translucent Materials}
    \author{Ben West \& Jonathan Wrona}
    % \thanks{Advisor: Dipl.--Ing.~Firstname Lastname, Lehrstuhl f\"ur Nachrichtentechnik, TUM, WS 2050/2051.}}
    \markboth{Advanced Computer Graphics - Spring 2015}{}
    \maketitle

\section{Summary of Technical Problem}
    \PARstart{F}{or} our project we plan to extend ray tracing to render materials with interesting physical properties. This will include some simpler goals such as capturing the mirror and distortion affects of glass objects, and contains the final goal of realistically rendering static fluids. To do this we will need to implement the refractive and reflective properties of transluscent materials, and a model for subsurface scattering. Once these are done we also hope to implement some features to allow us to extend this to more complicated scenes. These features may include an acceleration data structure, bouncing rays off of an interpolated normal so we can render triangle meshes well, placing an image behind the camera to be reflected, and rendering to an image file. We would also ideally like to use photon mapping to capture caustics.

\section{Relevant Research Papers}
    \subsection{A Practical Model for Subsurface Light Transport}
        This paper details how to implement a dipole model for subsurface scattering. This will be the technique that we will implement in our project in order to produce more realistic looking liquids that are not completely translucent.
    \subsection{Global Illumination using Photon Maps}
        In order to add an additional element of realism we hope to utilize photon mapping to generate caustics in our images. This paper explains the generation of the photon map, as well as a method for gathering the photons for ray tracing.
        \subsection{Correlated Multi-Jittered Sampling}
        This paper details a sampling method developed by pixar which provides random access to the samples without precomputation, and in which samples can be ordered or shuffled. We may implement this sampling method into our application if we have time remaining.
    \subsection{On building fast kd-Trees for Ray Tracing, and on doing that in O(N log N)}
        In order to increase the speed at which our ray tracer runs, we may implement the kd-Tree outlined in this paper, which was developed specifically for ray tracing. Throughout the paper a few different methods of building kd-Trees are mentioned, and then the improved version is presented and explained.
    \subsection{Distributed Ray Tracing}
        Last of all we may implemented the depth of field ray tracing technique which was explained in Pixar's distributed ray tracing paper. This addition would be for the added visual effect, with no significant addition to realism.

\section{Methods of Testing}
    \subsection{To test the reflection and refraction of transluscent materials we will use three scenes:}
        \begin{itemize}
            \item A non-perfect glass pane capturing the reflection of a sphere
            \item A hollow sphere distorting the scene behind it and capturing some of the scene in front of it
            \item A cup of water disorting scene behind it
        \end{itemize}
    \subsection{To test subsurface scattering we will use these scenes:}
        \begin{itemize}
            \item Spheres of various sizes to see if the sense of scale is captured
            \item Glasses of different milks
            \item A glass of some transluscent fluid such as wine
        \end{itemize}

\section{Assignment Timeline}
    \begin{enumerate}
        \item \textbf{Reflection \& Refraction} \emph{core}: Person.
        \item \textbf{Subsurface Scattering} \emph{core}: Person.
        \item \textbf{Photon Mapping} \emph{optional}: Person.
        \item \textbf{kd-Tree} \emph{optional}: Person.
        \item \textbf{Multi-Jittered Sampling} \emph{optional}: Person.
        \item \textbf{Normal Interpolation for Reflection} \emph{optional}: Person.
        \item \textbf{Rendering Directly to a File} \emph{optional}: Person.
        \item \textbf{Depth of Field} \emph{optional}: Person.
    \end{enumerate}
    Once the core features are complete, the initial goal is to complete feature number three. Then features four to seven can follow in any order, followed by feature eight.

\section{Conclusion}
    Our primary goals are to accurately capture the refractive and reflective properties of transluscent materials. Things we will work on given time include photon mapping to capture caustics through fluids, smooth renderings of triangle meshes, reflecting an image from behind the camera, getting a KD tree implemented for ray tracing, rendering to an image file, and lastly  capturing depth of field.

% Now we need a bibliography:
\begin{thebibliography}{5}

    % http://graphics.pixar.com/library/DistributedRayTracing/paper.pdf
    \bibitem{CPC84}
    Robert L Cook, Thomas Porter, Loren Carpenter: "Distributed Ray Tracing". Proceedings of SIGGRAPH'1984.

    % http://graphics.stanford.edu/papers/bssrdf/bssrdf.pdf
    \bibitem{JMLH01}
    Henrik Wann Jensen, Stephen R. Marschner, Marc Levoy and Pat Hanrahan: "A Practical Model for Subsurface Light Transport". Proceedings of SIGGRAPH'2001.

    % http://graphics.ucsd.edu/papers/layered/layered.pdf
    \bibitem{DJ05}
    Craig Donner, Henrik Wann Jensen: "Light Diffusion in Multi-Layered Translucent Materials". Proceedings of SIGGRAPH'2005.

    % http://graphics.ucsd.edu/~henrik/papers/ewr7/global_illumination_using_photon_maps_egwr96.pdf
    \bibitem{J96}
    Henrik Wann Jensen: "Global Illumination using Photon Maps". In "Rendering Techniques '96". Eds. X. Pueyo and P. Schröder. Springer-Verlag, pp. 21-30, 1996

    % http://graphics.pixar.com/library/MultiJitteredSampling/paper.pdf
    \bibitem{K13}
    Andrew Kensler: "Correlated Multi-Jittered Sampling". March 5, 2013.

    % http://dcgi.felk.cvut.cz/home/havran/ARTICLES/ingo06rtKdtree.pdf
    \bibitem{WH}
    Ingo Wald, Vlastimil Havran: "On buildilng fast kd-Trees for Ray Tracing, and on doing that in O(N log N)".

\end{thebibliography}

% Your document ends here!
\end{document}